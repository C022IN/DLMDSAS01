\documentclass{article}
\usepackage[hidelinks]{hyperref}
\usepackage{graphicx}    
\usepackage{wrapfig}
\usepackage{amsmath}
\usepackage{amssymb}
\newcommand\R{{\mathbb R} }
\renewcommand\S{{\mathcal S}}
\newcommand\goesto{{\longrightarrow}}

\begin{document}
\title{A simple $t$-test}
\date{2021-05-20}

\author{Paul Libbrecht, IUBH Advanced Statistics, \href{https://creativecommons.org/licenses/by/4.0/}{CC-BY}}
\maketitle



% =================================================

\section{Statement}

Every claims that the flags of the world are cut along a golden ratio, that is 

$$ \frac{width}{height} = \frac{1}{2} + \frac{\sqrt{5}}{2} \approx 1.618$$

... in average. Evaluate this statement with a signficance of $\alpha = 0.05$.

% ===================================================

\section{Data}

We fetch the data from \url{https://en.wikipedia.org/wiki/Gallery_of_sovereign_state_flags} (TODO: cite) using a little web-page 
script as follows:

\begin{verbatim}
f = (img) => { 
   let countryName = img.parentElement.parentElement.parentElement.parentElement.innerText; 
   if (countryName.startsWith("Flag of ")) countryName = countryName.substring(8); 
   let txt = '"' + countryName + '","' + img.width/img.height + '"\n'; 
   allText = allText + txt; console.log(txt)
}
var imgs = jQuery(".mw-parser-output img")
var allText = ""; imgs.each((n)=>{f(imgs[n]);})
\end{verbatim}

This obtains the values in the neighbour CSV file with $n=206$ flags and an average of about $1.67$ and standard deviation 
of $0.257$.

% ===================================================================

\section{Interpretation}

We make the hypothesis that the aspect ratio of the flags $R$ is distributed along a normal distribution $R\sim {\mathcal{N}}(1.618,\sigma)$. Where $\sigma$ is a standard-deviation that we do not know of.

We want to evaluate if the sample that we have follows a normal distribution with the same average with significance $0.01$
and formulate the hypotheses:
\begin{itemize}
\item $H_0$: $\mu_R \approx 1.618$
\item $H_1$: $\mu_R \not\approx 1.618$
\end{itemize}

We thus want to apply a two-tailed $t$-test TODO cite. e.g. \cite[theorem 7.4.2]{Larson-Marx} which will allow us
to reject the $H_0$ in favour of $H1$ if $t\leqslant -t_{\alpha/2,n-1}$ or $t\geqslant t_{\alpha/2,n-1}$


% ======================================================================

\section{Theory}

The student-$t$-theorem tells us that, considering $R_1, ..., R_n$ independent normal random variables, the following
variable follows a Student-$t$-distribution of $n-1$ degrees of freedom:

$$T_{n-1}=\frac{\overline{R}-\mu}{S_{R_1,...R_n}/\sqrt{n}}$$

where $S_{R_1,...R_n}$ is the sample standard deviation. TODO cite

Applying the $t$-test means testing if the probability of having this sample is at least $1-\alpha$.

% ========================================================================
\section{Computation}

$$t \approx \frac{1.67-1.618}{0.257\cdot\sqrt{206}} \approx 0.014$$

$$t_{\alpha/2,n-1} \approx 1.960$$

So $t$ is by far not less than $-t_{\alpha/2,n-1}$ or bigger than $t_{\alpha/2,n-1}$.



% ========================================================================
\section{Conclusion}

We are not able to conclude that the flags follow a different distribution.


\bibliographystyle{alpha}
\bibliography{a-simple-test}



\end{document}
